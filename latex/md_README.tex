\begin{quote}
Frog\+Race \end{quote}


Este repositório contém\+:


\begin{DoxyEnumerate}
\item Informações sobre a configuração
\item Instruções de compilação
\item Instruções de execução
\item Informações sobre localização dos códigos que serão avaliados
\end{DoxyEnumerate}

\subsection*{Tabela de Conteúdos}


\begin{DoxyItemize}
\item \href{#configuração}{\tt Configuração}
\item \href{#compilação}{\tt Compilação}
\item \href{#execução}{\tt Execução}
\item \href{#códigos-avaliados}{\tt Códigos Avaliados}
\begin{DoxyItemize}
\item \href{#passos-1-2}{\tt Passos 1, 2}
\item \href{#passos-3-4-5}{\tt Passos 3, 4 e 5}
\item \href{#passo-6}{\tt Passo 6}
\end{DoxyItemize}
\end{DoxyItemize}

\subsection*{Configuração}

Para o desenvolvimento deste projeto foi utilizado apenas as bibliotecas padrão do c++, $<$iostream$>$, $<$string$>$, $<$random$>$, $<$fstream$>$, $<$istream$>$ e $<$ostream$>$. Com excessão da T\+AD Lista utilizada, que foi implementada anteriormente e depois incluída no projeto com seu respectivo Iterador.

\subsection*{Compilação}

Para compilar o programa é necessário que existam as demais pastas do projeto. Caso elas não existam, execute o comando \textquotesingle{}make dir\textquotesingle{} a partir da pasta raíz do projeto. Também é necessária a existência dos arquivos corridas.\+csv e sapos.\+csv dentro da pasta data. Finalmente, basta compilar o programa executando o comando \textquotesingle{}make\textquotesingle{} ou \textquotesingle{}make Frog\+Race\textquotesingle{}.

\subsection*{Execução}

O programa ficará localizado na pasta bin. Para executá-\/lo a partir da pasta raíz do projeto, execute o comando \textquotesingle{}./bin/\+Frog\+Race\textquotesingle{}.

\subsection*{Códigos Avaliados}

Os passos 1, 2 e 6 estão localizados no arquivo \hyperlink{dataManager_8h}{data\+Manager.\+h}. Os passos 3, 4, 5 estão localizados ni arquivo \hyperlink{main_8cpp}{main.\+cpp}.

\subsection*{Passos 1, 2}

Linhas 76-\/79. Função que funciona para ambas classes \hyperlink{classSapo}{Sapo} e \hyperlink{classCorrida}{Corrida}.

\subsection*{Passos 3, 4 e 5}

Passo 3\+: Linhas 184-\/192 Descrevem a função que exibe o menu na tela. Linhas 204-\/233 Contém o loop lógico do menu principal. Passo 4\+: Linhas 152-\/168 Descrevem a função que retorna a escolha da corrida. Passo 5\+: Linhas 95-\/135 Descrevem a função que determina a lógica da corrida propriamente dita.

\subsection*{Passo 6}

Criar o \hyperlink{classSapo}{Sapo}\+: Linhas 128-\/137. Criar a \hyperlink{classCorrida}{Corrida}\+: Linhas 146-\/163. Salvar \hyperlink{classSapo}{Sapo}\+: Linhas 108-\/119. 